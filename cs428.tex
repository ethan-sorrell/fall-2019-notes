% Created 2019-09-05 Thu 20:27
% Intended LaTeX compiler: pdflatex
\documentclass[11pt]{article}
\usepackage[utf8]{inputenc}
\usepackage[T1]{fontenc}
\usepackage{graphicx}
\usepackage{grffile}
\usepackage{longtable}
\usepackage{wrapfig}
\usepackage{rotating}
\usepackage[normalem]{ulem}
\usepackage{amsmath}
\usepackage{textcomp}
\usepackage{amssymb}
\usepackage{capt-of}
\usepackage{hyperref}
\author{ethan}
\date{\today}
\title{}
\hypersetup{
 pdfauthor={ethan},
 pdftitle={},
 pdfkeywords={},
 pdfsubject={},
 pdfcreator={Emacs 26.1 (Org mode 9.2.5)}, 
 pdflang={English}}
\begin{document}

\tableofcontents

\section{Chapter 1: Overview}
\label{sec:orga1c0d61}
\subsection{The CIA Triad}
\label{sec:orga81b7a9}
\begin{itemize}
\item Confidentiality
\begin{itemize}
\item Preserving intended restrictions on information access
\end{itemize}
\item Integrity
\begin{itemize}
\item Guarding against improper information modification or destruction
\end{itemize}
\item Availability
\begin{itemize}
\item Ensuring timely and reliable access to information
\end{itemize}
\end{itemize}
\subsection{Additional Security Concepts}
\label{sec:orgf8a303b}
\begin{itemize}
\item Authenticity
\item Accountability
\end{itemize}
\subsection{Challenges}
\label{sec:orgddb815a}
\begin{itemize}
\item Only one vulnerability needs to be found by attacker,
defender must patch all vulnerabilities
\end{itemize}
\subsection{Categories of Vulnerabilities}
\label{sec:org96f35cd}
\begin{itemize}
\item Corrupted (loss of integrity)
\item Leaky (loss of confidentiality)
\item Unavailable (loss of availability)
\end{itemize}
\subsection{Categories of Attacks}
\label{sec:orga6a9e71}
\begin{itemize}
\item Passive
\begin{itemize}
\item attempt to learn or make use of information from the system that does not affect system resources
\end{itemize}
\item Active
\begin{itemize}
\item attempt to alter system resources or affect their operation
\end{itemize}
\item Insider
\begin{itemize}
\item initiated by an entity within the security perimeter
\end{itemize}
\item Outsider
\begin{itemize}
\item initiated by an entity outside of the security perimeter
\end{itemize}
\end{itemize}
\subsection{Security Functional Requirements}
\label{sec:orgb87ceb6}
\begin{itemize}
\item Security technical measures
\begin{itemize}
\item required to deal with access control problems
\item identification and authentication
\item system and communication protection
\item system and information integrity
\end{itemize}
\item Management controls and procedures
\begin{itemize}
\item awareness and training
\item audit and accountability
\item certification, acreditation and security assessments
\end{itemize}
\end{itemize}
\section{Chapter 2: Cryptographic Tools}
\label{sec:orgbe22047}
\subsection{Cryptography vs. Security}
\label{sec:org6a5db6d}
\begin{itemize}
\item Theory vs Practice
\item Tools vs. Applications
\item etc
\end{itemize}
\subsection{Symmetric Encryption}
\label{sec:orgea1b2f8}
\begin{itemize}
\item Also referred to as conventional encryption or single-key encryption
\end{itemize}
\subsubsection{Two requirements for secure use:}
\label{sec:org80e0abc}
\begin{itemize}
\item Strong encryption algorithm
\item Sender and receiver must have obtained copies of the secret key in a secure fashion
\end{itemize}
\subsection{Attacking Symmetric Encryption}
\label{sec:org4504a48}
\begin{itemize}
\item Broadly divided into cryptanalytic and brute-force attacks
\item Crypt analytic exploits weaknesses in the algorithm
\item A more critical possibility:
\begin{itemize}
\item Stealing the keys from memory or disk
\end{itemize}
\end{itemize}
\subsection{Table 2.1}
\label{sec:orgd38a9af}
\begin{center}
\begin{tabular}{lrrr}
 & DES & Triple DES & AES\\
\hline
Plaintext block size (bits) & 64 & 64 & 128\\
Ciphertext block size (bits) & 64 & 64 & 128\\
Key size (bits) & 56 & 112 or 168 & 128, 192, or 256\\
\hline
\end{tabular}
\end{center}
\subsection{Data Encryption Standard (DES)}
\label{sec:org1fa8f28}
\begin{itemize}
\item Most widely used encryption scheme
\item Uses 64 bit plaintext block and 56 bit key to produce 64 bit ciphertext block
\item Use of 56-bit key is strength concern:
\begin{itemize}
\item EEF announced in July 1998 it had broken a DES encryption
\end{itemize}
\item See diagram in slides
\end{itemize}
\subsubsection{Triple DES (3DES)}
\label{sec:orgb3d7d12}
\begin{itemize}
\item Repeat DES algorithm three times
\item First standardized for use in finance in ANSI standard X9.17 in 1985
\item Sluggish performance
\end{itemize}
\subsection{Advanced Encryption Standard (AES)}
\label{sec:org4ebd740}
\begin{itemize}
\item NIST called for proposals for a new AES in 1997
\item Selected Rijndael in November 2001
\item See diagram in slides
\end{itemize}
\subsection{Practical Security Issues}
\label{sec:orgb33ff6e}
\begin{itemize}
\item Typically encryption is applied to a unit  of data larger than e.g. a single 64-bit block
\item Electronic codebook (ECB) mode is the simplest approach to multiple-block encryption
\begin{itemize}
\item Each block is encrypted using the same key
\item Regularities in the plaintext may be exploitable
\end{itemize}
\item Modes of operation
\begin{itemize}
\item Alternative technique developed to increase security for symmetric block encryption of large sequences
\item Overcomes the weaknesses of ECB
\end{itemize}
\end{itemize}
\subsection{Block Ciphers}
\label{sec:org312c147}
\begin{itemize}
\item Processes the input one block of elements at a time
\item Produces separate output block for each input block
\item Can reuse keys
\item More common (than stream cipher?)
\end{itemize}
\subsection{Stream Ciphers}
\label{sec:orgb6951af}
\begin{itemize}
\item Processes the input elements continuously
\item Primary advantage is that this form of cipher is faster and uses less code
\item Pseudorandom stream is one that is unpredictable without knowledge of the input key
\end{itemize}
\subsection{Public-Key Encryption Structure}
\label{sec:orgb034db7}
\begin{itemize}
\item Asymmetric
\begin{itemize}
\item Uses two separate keys
\item Public key and private key
\item Some form of protocol is needed for distribution
\end{itemize}
\item See slides for diagram
\item Each user has a public and private key
\item Use intended receiver's public key to encrypt, They use their private key to decrypt
\item Or the opposite
\item e.g. PGP is Public Key Encryption
\end{itemize}
\subsection{RSA}
\label{sec:org199b041}
\begin{itemize}
\item to encrypt a message M, the sender:
\begin{itemize}
\item obtains public key of recipient: \{e, n\}
\item computes: C = M\textsuperscript{e} mod n where 0 <= M < n
\end{itemize}
\item to decrypt the ciphertext C, the owner:
\begin{itemize}
\item uses their private key: \{d, n\}
\item computes: M = C\textsuperscript{d} mod n
\end{itemize}
\end{itemize}
\subsection{Requirements for Public-Key Cryptosystems}
\label{sec:org344598f}
\begin{itemize}
\item Computationally easy to:
\begin{itemize}
\item create key pairs
\item encrypt message given public key
\item decrypt ciphertext knowing private key
\end{itemize}
\item Computationally infeasible to:
\begin{itemize}
\item recover original message without private key
\item determine private key from public key
\end{itemize}
\item Also useful if either key can be used for either role
\end{itemize}
\subsection{Asymmetric Encryption Algorithms}
\label{sec:orgf411ed8}
\begin{itemize}
\item RSA (Rivest, Shamir, Adleman)
\begin{itemize}
\item Developed in 1977
\item Most widely accepted and implemented approach to public-key encryption
\item Block cipher in which the plaintext and ciphertext are integers between 0 and n-1
\end{itemize}
\item Diffie-Hellman key exchange
\begin{itemize}
\item Enables two users to securely reach agreement about a shared secret that can be used as a secret key for subsequent symmetric encryption of messages
\item Limited to the exchange of keys
\end{itemize}
\item Digital Signature Standard (DSS)
\begin{itemize}
\item Provides only a digital signature function with SHA-1
\item Cannot be used for encryption or key exchange
\end{itemize}
\item Elliptic Curve Cryptography (ECC)
\begin{itemize}
\item Security like RSA, but with much smaller keys
\end{itemize}
\end{itemize}
\subsection{Table 2.3}
\label{sec:orge77a887}
\begin{center}
\begin{tabular}{llll}
Algorithm & Digital Signature & Symmetric Key Distribution & Encryption of Secret Keys\\
\hline
RSA & Yes & Yes & Yes\\
Diffie-Hellman & No & Yes & No\\
DSS & Yes & No & No\\
Elliptic Curve & Yes & Yes & Yes\\
\end{tabular}
\end{center}
\subsection{Symmetric vs Public Key}
\label{sec:org26d4948}
\begin{itemize}
\item Conventional Encryption
\end{itemize}
\subsection{Message Authentication}
\label{sec:org6379134}
\subsection{Hash Function Requirements}
\label{sec:orgce7c9e0}
\begin{itemize}
\item Can be applied to any size block of data
\item Produces a fixed-length output
\item Relatively easy to compute for any input
\item One way or pre-image resistant (not reversible to get input)
\item Second pre-image resistant or weak collision resistant
\begin{itemize}
\item Computationally infeasible to find input which will hash to the same value as given input
\end{itemize}
\item Collision resistant or strong collision resistance
\begin{itemize}
\item Computationally infeasible to find any pair of inputs that will hash to the same value
\end{itemize}
\end{itemize}
Note the distinction between weak collision resistant and strong collision resistant 
\subsection{Security of Hash Functions}
\label{sec:org8272a83}
\begin{itemize}
\item SHA most widely used hash algorithm
\end{itemize}
\subsubsection{Additional secure hash function applications:}
\label{sec:orge0f7048}
\begin{itemize}
\item Password
\begin{itemize}
\item Hash of a password is stored rather than password itself
\end{itemize}
\item Intrusion detection
\begin{itemize}
\item Store H(F) for each file on a system and secure the hash values
\item Now possible to detect manipulation by recomputing hash values
\end{itemize}
\end{itemize}
\subsubsection{Two approaches to attacking a secure hash function}
\label{sec:orgf9b79f1}
\begin{itemize}
\item Cryptanalysis
\item Brute-force
\end{itemize}
\subsection{Digital Signatures}
\label{sec:orgc65b0e8}
\begin{itemize}
\item Think PGP
\end{itemize}
\subsection{Digital Envelopes}
\label{sec:org793618c}
\begin{itemize}
\item Rather than using public key to encrypt message, encrypt a symmetric key using public key, and send message encrypted with this symmetric key
\item This is useful as symmetric key encryption is quicker
\end{itemize}
\subsection{Random Numbers Uses}
\label{sec:orga1923fe}
\begin{itemize}
\item Keys for public-key algorithms
\item Stream key for symmetric stream cipher
\item Symmetric key for use in e.g. creating a digital envelope
\item Handshaking to prevent replay attack
\item Session key
\end{itemize}
\subsection{Practical Application: Encryption of Stored Data}
\label{sec:org98bdd74}
\begin{itemize}
\item Common to encrypt transmitted data
\item Much less common to encrypt stored data
\end{itemize}
\section{Chapter 10: Buffer Overflow}
\label{sec:orgee092af}
\subsection{Buffer Overflow}
\label{sec:org0c418cb}
\begin{itemize}
\item A very common attack mechanism
\item Known prevention techniques
\item A buffer overflow, also known as buffer overrun defined:
\begin{itemize}
\item A condition at an interface under which more input can be placed into a buffer than the capacity allocated, overwriting other information. Exploited to crash a system, or insert carefully crafted code to gain control of the system.
\end{itemize}
\end{itemize}
\subsection{Buffer Overflow Basics}
\label{sec:org2629f9c}
\begin{itemize}
\item A process attempts to store data beyond the limits of a fixed-sized buffer
\item Overwrites adjacent memory locations
\item Buffer could be located on the stack, in the heap, or in the data section of the process
\end{itemize}
\subsection{Programs and Processes}
\label{sec:org309c5d3}
\begin{center}
\begin{tabular}{l}
Kernel Code and Data\\
Stack\\
Spare Memory\\
Heap\\
Global Data\\
Program Machine Code\\
Process Control Block\\
\end{tabular}
\end{center}
\subsection{Buffer Overflow Attacks}
\label{sec:org14872e3}
\begin{itemize}
\item The exploit a buffer overflow an attacker needs:
\begin{itemize}
\item To identify a buffer overflow vulnerability in some program that can be triggered using data under the attacker's control
\item To understand how that buffer is stored in memory and determine potential for corruption
\end{itemize}
\item Identifying vulnerable programs can be done by:
\begin{itemize}
\item Inspection of program source
\item Tracing the execution of programs as they process over-sized input
\item Using tools such as fuzzing to automatically identify potential vulnerabilities
\end{itemize}
\end{itemize}
\subsection{Programming Language History}
\label{sec:orga46b67a}
\begin{itemize}
\item At the machine level, data manipulated by machine instructions much be stored in either the processor's registers or in memory
\item Assembly language program is responsible for the correct interpretation of any saved data value
\end{itemize}
\subsection{Stack Buffer Overflows}
\label{sec:orgff628e7}
\begin{itemize}
\item Occur when overflown buffer is located on stack
\begin{itemize}
\item Also referred to as stack smashing
\end{itemize}
\item Still being widely exploited
\item Stack frame
\begin{itemize}
\item Stores return address when one function calls another function
\item Also needs location to save passed parameters and possible save register values
\end{itemize}
\end{itemize}
\end{document}
